\documentclass[reqno,T1]{amsproc}
\usepackage{amssymb}
\usepackage{amsmath}
\usepackage{amsfonts}
\usepackage{multicol}
\usepackage[canadian]{babel}
\usepackage{xcolor}
\usepackage{geometry}
\usepackage{xifthen} % provides \isempty
\usepackage{xparse} % new commands with more than one optional argument
\usepackage{fixltx2e} % for \textsubscript
\newcommand{\unlim}{\qopname\relax m{unlim}}
\newcommand{\uwlim}{\qopname\relax m{uwlim}}

% commutative diagrams
\usepackage{tikz}
\usetikzlibrary{arrows,cd,decorations.markings,shapes.geometric,shapes}

% color of links
\definecolor{myurlcolor}{rgb}{0,0,0.4}
\definecolor{mycitecolor}{rgb}{0,0.5,0}
\definecolor{myrefcolor}{rgb}{0.5,0,0}
\usepackage[pagebackref,draft=false]{hyperref}
\hypersetup{colorlinks,
linkcolor=myrefcolor,
citecolor=mycitecolor,
urlcolor=myurlcolor}
\renewcommand*{\backref}[1]{$\uparrow$\,#1}

% general mathematical macros
\renewcommand{\H}{\mathcal{H}}	% Hilbert space
\newcommand{\beq}{\begin{equation}}
\newcommand{\eeq}{\end{equation}}
\newcommand{\N}{\mathbb{N}}
\newcommand{\Z}{\mathbb{Z}}
\newcommand{\Q}{\mathbb{Q}}
\newcommand{\R}{\mathbb{R}}
\newcommand{\Rplusext}{\overline{\mathbb{R}}_+} % extended nonnegative reals, as where measures take values
\newcommand{\C}{\mathbb{C}}
\newcommand{\op}{\mathrm{op}}
\newcommand{\eps}{\varepsilon}
\newcommand{\lin}[1]{\mathrm{lin}(#1)}
\newcommand{\dom}[1]{\mathrm{dom}(#1)} 		% domain of a morphism
\newcommand{\cod}[1]{\mathrm{cod}(#1)} 		% codomain of a morphism
\newcommand{\supp}{\mathrm{supp}}
\newcommand{\esssup}{\mathrm{ess\, sup}} 	% essential supremum

% categories
\newcommand{\cat}[1]{\mathsf{#1}}		% generic category
\newcommand{\Set}{\mathsf{Set}}
\newcommand{\Vect}[1]{\mathsf{Vect}_{#1}}	% vector spaces over a field
\newcommand{\Ban}{\mathsf{Ban}}			% (complex) Banach spaces
\newcommand{\Hilb}{\mathsf{Hilb}} 		% Hilb, cat of Hilbert spaces
\newcommand{\Rep}[1]{\mathsf{Rep}(#1)}		% cat normal Hilbert space reps of a C$^*$-algebra
\newcommand{\NRep}[1]{\mathsf{nRep}(#1)}	% cat of normal Hilbert space reps of a W$^*$-algebra
\newcommand{\id}[1]{1_{#1}}			% identity morphism
\newcommand{\Kay}{\mathcal{K}} % I use a convention that the name of a letter spelt phonetically signifies its mathcal version. But semantically, this is a compact class in a measure space.
\newcommand{\powerset}{\mathcal{P}}

% Theorem Environments
\swapnumbers
\theoremstyle{plain}
\newtheorem{thm}{Theorem}[section]
\newtheorem{lem}[thm]{Lemma}
\newtheorem{prop}[thm]{Proposition}
\newtheorem{cor}[thm]{Corollary}
\newtheorem{conj}[thm]{Conjecture}
\newtheorem{qstn}[thm]{Question}
\newtheorem{defn}[thm]{Definition}
\newtheorem{lemdefn}[thm]{Lemma + Definition}
\newtheorem{prob}[thm]{Problem}
\theoremstyle{remark}
\newtheorem{ex}[thm]{Example}
\newtheorem{rem}[thm]{Remark}
\newtheorem{note}[thm]{Note}
\numberwithin{equation}{section}

%%%%%%%%%%%% Enumeration via lowercase letters
\renewcommand{\labelenumi}{(\alph{enumi})}
\renewcommand{\theenumi}{(\alph{enumi})}
\renewcommand{\labelitemi}{$\circ$}

%%% pagebreaks allowed for align environment
\allowdisplaybreaks

% string diagram stuff
\pgfdeclarelayer{edgelayer}
\pgfdeclarelayer{nodelayer}
\pgfsetlayers{edgelayer,nodelayer,main}
\tikzstyle{none}=[inner sep=0pt]
\tikzstyle{simple}=[-,draw=black,line width=1.000]
\tikzstyle{bn}=[circle,inner sep=2pt,fill=black,draw=black,line width=0.8 pt]

%%% draft stuff
%\usepackage[draft]{showkeys}
\usepackage{todonotes}
\newcommand{\tob}[1]{\todo[color=blue!40,inline]{\textbf{T}: #1}}


%-------------------------------------------------------------------

\begin{document}
\sloppy

% vertical spacing in multiline equations
\setlength{\jot}{6pt}

%-------------------------------------------------------------------

%%%%%%%%%%%% title page stuff %%%%%%%%%%%%%%%%%%%%%%%%%%

\title{The universal property of direct integrals in W$^*$-categories}

\author{Tobias Fritz, Robert Furber, Bas Westerbaan}

\address{Max Planck Institute for Mathematics in the Sciences, Leipzig, Germany}
\email{fritz@mis.mpg.de}

\keywords{}

\subjclass[2010]{Primary: ; Secondary: }

\thanks{\textit{Acknowledgements.} }

\begin{abstract}
Using enrichment in Banach spaces, we observe that one can formulate universal properties characterizing universal objects in C$^*$-categories in such a way that they guarantee uniqueness up to unique unitaries. We then formulate such a universal property for infinite direct sums in W$^*$-categories, and prove that this indeed characterizes the standard infinite direct sums in the W$^*$-category of normal representations of a W$^*$-algebra. We do the same for direct integrals, in which case we need to work with measurable W$^*$-categories.
\end{abstract}

\maketitle

\section{Introduction}


\section{C$^*$-categories}

For this section, we work in a C$^*$-category \cite{wstarcat} denoted by $\cat{C}$.
We start with a generalization of~\cite[Lemma 7]{westerbaan2016universal}
    to C$^*$-categories.

\begin{lem}
For any~$a\colon A \to B$ with~$\|a\| \leq 1$
    and projections~$s \colon A \to A$,
        $t\colon B\to B$,
	the following are equivalent:
\begin{enumerate}
\item $a^* t a \leq 1-s$,
\item $a s a^* \leq 1 - t$,
\item $tas = 0$,
\item $sa^* t = 0$.
\end{enumerate}
\label{contrapositionlemma}
\end{lem}
\begin{proof}
We will first prove~$a^*t a \leq 1-s \Leftrightarrow tas=0$.
    Assume~$a^* t a \leq 1-s$.
    Then~$0 \leq s a^* t a s \leq s (1-s) s = 0$.
    Thus~$s a^* t a s = 0$, which is to say~$tas = 0$.
    Now, assume~$tas = 0$.
    Then~$ta = ta(1-s)$ and~$a^*t = (1-s)a^* t$.
    Hence~$a^* ta = (1-s)a^*ta(1-s) \leq 1-s$, as desired.
A similar argument proves~$as a^* \leq 1-t \Leftrightarrow sa^*t=0$.
To finish the proof, it is sufficient to
        show~$tas=0 \Leftrightarrow sa^*t=0$, which follows directly
        by applying~$(\ )^*$.
\end{proof}

\begin{lem}
Let~$a \colon A \leftrightarrows B \colon b$ be any two morphisms
    such that~$\|a \| \leq 1$, $\| b\| \leq 1$ and~$ab=1$.
    Then~$b$ is an isometry with~$a=b^*$.
\label{isometrylemma}
\end{lem}
\begin{proof}
    To start, note~$\|a ^*a\| = \|a\|^2 \leq 1$, so~$a^*a \leq 1$.
    Similarly~$b^*b \leq 1$.
    Combined~$1 = b^*a^*ab \leq b^*b \leq 1$, hence~$b^*b=1$.
    Similarly~$aa^*=1$.
    Clearly~$bb^*bb^* = b 1 b^* = bb^*$, so~$b$ is an isometry.
    Now~$a (1-bb^*) a^* = aa^* - abb^*a^* = 1-1=0$.
    So~$a^* (1-0) a \leq bb^*$ by Lemma~\ref{contrapositionlemma}.
    Similarly~$bb^* \leq a^*a$. Hence~$bb^*=a^*a$.
    We are done: $a = aa^*a=abb^*=b^*$.
\end{proof}

We will need the full power of this observation only in the next section. However, the following special case is generally useful for the definition of universal properties in C$^*$-categories:

\begin{cor}
Let $u : A \to B$ be an invertible morphism with $\| u \| \leq 1$ and $\| u^{-1} \| \leq 1$. Then $u$ is unitary.
\label{unitaries}
\end{cor}

Now let $\Ban$ be the category of Banach spaces with bounded linear maps of norm $\leq 1$. Every C$^*$-category is $\Ban$-enriched. Lemma~\ref{unitaries} and the Yoneda lemma imply that for any representable functor $\cat{C} \to \Ban$, the representing object is unique up to unique \emph{unitary} isomorphism. This is a desideratum of a notion of universal property for C$^*$-categories, and more generally for all dagger categories~\cite{daglims}. The lemma tells us that the compatibility with the dagger structure is automatic thanks to the enrichment in $\Ban$, so that we do not need to formulate our universal properties in the form of dagger limits as in~\cite{daglims}.

In summary, our notion of universal property in a C$^*$-category is the standard one for enriched categories, namely with enrichment in $\Ban$:

\begin{defn}
A functor $F : \cat{C} \to \Ban$ is \emph{representable} if there is an object $A \in \cat{C}$ together with a natural isomorphism $\cat{C}(A,-)\cong F$ that is a componentwise isometric isomorphism of Banach spaces.
\end{defn}

Moreover, thanks to Corollary~\ref{unitaries}, the representing object is unique up to a unique unitary: if both $A$ and $B$ represent $F$ via $\cat{C}(A,-)\cong F\cong \cat{C}(B,-)$, then the Yoneda lemma gives us an isomorphism $u : A \to B$. Since the natural isomorphism between the hom-functors is required to be a componentwise isometry, we have $\| u \| = \| u^{-1} \| = 1$, making $u$ into a unitary.

\section{The universal property of infinite direct sums}

We now turn to a particular universal property and study infinite direct sums. (While our considerations also apply to finite direct sums, in this case our discussion provides nothing new, since then it reduces to the well-understood case of biproducts in an additive category.) From now on, we take $C$ to be a W$^*$-category~\cite{wstarcat}, such as $\Hilb$ or $\Rep{A}$, the category of Hilbert space representations of a C$^*$-algebra $A$.

For $(A_i)_{i\in I}$ a family of objects in $\cat{C}$, we consider the functor
\[
	\bigoplus_i \cat{C}(A_i,-) \: : \: \cat{C} \to \Ban
\]
given by associating to every object $B$ the vector space
\[
	\bigoplus_i \cat{C}(A_i , B) := \left\{ (f_i : A_i \to B)_{i\in I} \Biggm\vert \sum_i f_i f_i^* < \infty \right\},
\]
considered as a Banach space under the norm
\[
	\| (f_i)_{i \in I} \| := \left\| \sum_i f_i f_i^* \right\|^{1/2}.
\]
This is functorial in $B$ in the obvious way.

\begin{lem}
$\bigoplus_i \cat{C}(A_i,-)$ is complete, and therefore a Banach space.
\end{lem}

The proof is analogous to the standard proof showing that $\ell^2(I)$ is a Hilbert space (and generalizes it).

\begin{proof}
Suppose that we are given a Cauchy sequence of families $(f^{(n)})_{n\in\N} = ( ( f_i^{(n)} )_{i\in I} )_{n\in\N}$. For given $i$, projecting to $\cat{C}(A_i,B)$ is bounded, and therefore $( f_i^{(n)} )_{n\in\N}$ is Cauchy as well; let us denote the limit by $f_i : A_i \to B$. We will now show that $f \in \oplus_i \cat{C}(A_i,B)$ for $f := (f_i)_{i\in I}$. The Cauchy assumption implies that the sequence $(f^{(n)})_{n\in\N}$ is uniformly bounded by some constant $C$,
\[
	\left\| \sum_i (f_i^{(n)})^* f_i^{(n)} \right\|^{1/2} \leq C.
\]
Restricting to a finite partial sum over $S\subseteq I$ and taking $n\to\infty$ shows that
\[
	\left| \sum_{i\in S} f_i^* f_i \right\|^{1/2} \leq C
\]
as well. Since $S$ was arbitrary, we conclude that $\| f \| \leq C$.


The convergence $f_i \to f$ works similarly. By assumption, for every $\eps > 0$ we have $\left\| f^{(n)} - f^{(m)} \right\| \leq \eps$ for almost all $n$ and $m$. In particular, we know that for any finite $S \subseteq I$,
\[
	\left\| \sum_{i\in S} (f_i^{(n)} - f_i^{(m)})^* (f_i^{(n)} - f_i^{(m)}) \right\|^{1/2} \leq \eps
\]
for almost all $n$ and $m$. Taking $m\to\infty$ gives
\[
	\left\| \sum_{i\in S} (f_i^{(n)} - f_i)^* (f_i^{(n)} - f_i) \right\|^{1/2} \leq \eps.
\]
Since $S$ was arbitrary, we conclude $\| f^{(n)} - f \| \leq \eps$ for almost all $n$.
\end{proof}

We now get to our main definition:

\begin{defn}
\label{directsumdef}
An \emph{$I$-indexed direct sum} of the family of objects $(A_i)_{i\in I}$ is an object $\bigoplus_i A_i$ which represents the $\Ban$-enriched functor $\bigoplus_i \cat{C}(A_i,-)$.
\end{defn}

By the discussion of the previous section, this characterizes infinite direct sums up to unique unitaries. Since we are in a dagger category, the same holds dually, meaning that $\bigoplus_i A_i$ also represents the functor
\[
	\bigoplus_i \cat{C}(-,A_i) \: : \: \cat{C}^\op \to \Ban
\]
given by associating to every object $B$ the vector space
\[
	\bigoplus_i \cat{C}(B, A_i) \ := \ \bigl\{\, (f_i : B \to A_i)_{i\in I} \bigm\vert \sum_i f_i^* f_i < \infty \,\bigr\},
\]
where the sum is taken in the ultraweak topology\footnote{
    For positive elements~$(x_i)_{i \in I}$ of a W$^*$-algebra,
        convergence and value
        of the sum~$\sum_{i \in I} x_i $
        coincides in the ultraweak and ultrastrong topology
        and, in fact, equals the supremum of the partial sums.}
    and where we use the norm
\[
	\| (f_i)_{i \in I} \|\  := \ \Bigl\| \,\sum_i f_i^* f_i \,\Bigr\|^{1/2},
\]
which is functorial in $B$ in the obvious way.

\begin{thm}
The following are equivalent for an object $A$:
\begin{enumerate}
\item\label{universalprop} $A$ is an $I$-indexed direct sum $\bigoplus_i A_i$.
\item\label{universalprop2} 
There is a family of
        morphisms~$\left(\kappa_j : A_j \to A\right)_{j\in I}$
        with~$\sum_{j \in I} \kappa_j \kappa_j^* < \infty$
        such that for any object~$B$
        and family of
        morphisms~$\left(f_j: A_j \to B\right)_{j\in I}$
        with~$\sum_{j \in I}f_j f_j^* < \infty$,
        there is a unique~$f\colon A \to B$
        such that $f \kappa_j = f_j$ for all~$j \in I$, and $\| f \|^2 = \| \sum_{j\in I} f_j f_j^* \|$.
\item\label{biprodsum} There is a family of morphisms $\left(\kappa_j : A_j \to A\right)_{j\in I}$ such that $\kappa_i^*\kappa_j = \delta_{ij}$, and
\beq
\label{complete}
        \sum_{j\in I} \kappa_j \kappa_j^* \ =\  1_A.
\eeq
\end{enumerate}
\end{thm}

One can also consider each of these three statements as a structure rather than a mere property, and then the proof establishes a bijection between these structures.

As far as we know, the first definition of infinite direct sums has been given in~\cite[p.~100]{wstarcat} in the form~\ref{biprodsum}. From our perspective, it is preferable to replace this definition by a universal property as in Definition~\ref{directsumdef}, and regard the original definition as an alternative characterization instead.

\begin{proof}
\ref{universalprop2} is a simple reformulation of the universal property \ref{universalprop} via the Yoneda lemma, as follows. Given a natural isometric isomorphism $\cat{C}(A,B) \cong \bigoplus_i \cat{C}(A_i,B)$, we define the family $(\kappa_i)$ to be the image of $\id{A}$. Then for any $f : A \to B$, the naturality diagram
\[\begin{tikzcd}
	\cat{C}(A,A) \ar{r}{\cong} \ar{d}{f\circ -} & \bigoplus_i \cat{C}(A_i,A) \ar{d}{(f\circ -)} \\
	\cat{C}(A,B) \ar{r}{\cong} & \bigoplus_i \cat{C}(A_i,B)
\end{tikzcd}\]
shows that the universal property takes $f$ to the family $(f_j) := (f\kappa_j)$. By the definition of the norm on $\bigoplus_i \cat{C}(A_i,B)$ and the assumption that the universal property is an isometric isomorphism, we indeed have $\| f \|^2 = \left\| \sum_j f_j f_j^* \right\|$, which also gives $\sum_j \kappa_j \kappa_j^* < \infty$ in the case $f = \id{A}$. Conversely, it is clear that any family $(\kappa_j)$ as in~\ref{universalprop2} gives an isometric isomorphism $\cat{C}(A,B) \cong \bigoplus_i \cat{C}(A_i,B)$ that is natural in $B$.

We now derive \ref{biprodsum} from \ref{universalprop2}. Applying the assumption on norms with $(f_j) = (\kappa_j)$ gives $\| \sum_i \kappa_i \kappa_i^* \| = \| \id{A} \|$, and therefore $\| \kappa_j \| \leq 1$. For fixed $j\in J$, the family $(\delta_{ij} : A_i \to A_j)$ corresponds to some morphism $\pi_j : A \to A_j$ such that $\pi_j\circ\kappa_i = \delta_{ij}$. Again by the isometry property, we have $\| \pi_j \| = \| \id{A} \| \leq 1$. By Lemma~\ref{isometrylemma}, this implies $\pi_j = \kappa_j^*$. We therefore have $\kappa_j^*\kappa_i = \delta_{ij}$, as desired. This also implies that the $\kappa_j \kappa_j^*$ are mutually orthogonal projections. In order to also prove the completeness relation~\eqref{complete}, let $p : A\to A$ be any other projection that is orthogonal to each of the $\kappa_j \kappa_j^*$; we need to show $p = 0$. Again by naturality, under the isomorphism $\cat{C}(A,A) \cong \bigoplus_i \cat{C}(A_i,A)$, the new $p$ corresponds to the family $(p\kappa_i) = (0)$, where $p\kappa_i = 0$ follows from the assumed orthogonality. This implies indeed $p=0$ by the uniqueness part of the universal property.

We finally show that \ref{biprodsum} implies \ref{universalprop2}. So let~$\left(f_j: A_j \to B\right)_{j\in I}$ be any family of morphisms
        with~$\sum_{j \in I}f_jf_j^* < \infty$.
The homset~$C(B,A)$
    is a self-dual Hilbert~$\cat{C}(B,B)$-module
    with~$\cat{C}(B,B)$-valued inner
    product~$\langle g, f\rangle \equiv g^*f$~\cite[2.15]{wstarcat}.
The self-duality is equivalent to the fact
    that~$\cat{C}(B,A)$ is complete in the \emph{ultranorm topology};
    this is the topology
    generated by the seminorms~$\| f \|_\omega \equiv \omega(\langle f,f\rangle)^\frac{1}{2}$
    indexed by normal states~$\omega$ on~$\cat{C}(B,B)$~\cite[\S149\textsubscript{V}]{bas}. We now turn to the construction of $f : A \to B$.
    Pick any normal state~$\omega$ on~$\cat{C}(B,B)$
        and finite subset~$S \subseteq I$.  We have
\begin{alignat*}{2}
    \bigl\| \sum_{j \in S} \kappa_j f_j^* \bigr\|^2_\omega
    &\ \equiv \ 
    \omega\Bigl(\bigl(\sum_{i \in S} \kappa_i f_i^*\bigr)^*
            \bigl(\sum_{j \in S} \kappa_j f_j^*\bigr)\Bigr) \\
    &\ =\ 
    \omega\Bigl( \sum_{j \in S} f_j f_j^*
    \Bigr) &\qquad&\text{as~$\kappa_j^*\kappa_i = \delta_{ij}$}.
\end{alignat*}
From this and~$\sum_{j \in I} f_jf_j^* < \infty$,
    it follows that the net~$\left(\sum_{j \in S} \kappa_j f_j^*\right)_S$
    is ultranorm Cauchy as $S$ ranges over all finite subsets of $I$.
    Define~$f := \left( \sum_{j \in I} \kappa_jf_j^* \right)^*$,
        where the sum is understood to converge ultranorm.
The assignment~$f \mapsto \kappa_j^*f$
    yields a bounded~$\cat{C}(B,B)$-linear map between
    the Hilbert~$\cat{C}(B,B)$-modules~$\cat{C}(B,A)$ and~$\cat{C}(B,A_j)$
    and is therefore ultranorm continuous~\cite[\S148]{bas}.
    Hence
        $\kappa_j^* f^*
             =  \kappa_j^* \sum_{i} \kappa_i f_i^*
             =  \sum_i \kappa_{j }^*\kappa_i f_i^*
            =  f_j^*$
            and so~$f \kappa_j = f_j$ as desired.
Next we will show that~$\|f \|^2 = \|\sum_j f_j f_j^*\|$.
    By definition of~$f^*$ and \cite[\S148\textsubscript{V}]{bas}
         we have~
 \begin{equation*}
         \langle f^*, f^*\rangle
         \ =\  \uwlim_{S\subseteq I\text{ finite}} \, \Bigl\langle \sum_{j \in S} \kappa_j f^*_j,
                        \sum_{j \in S}\kappa_j f^*_j \Bigr\rangle
         \ =\  \uwlim_{S\subseteq I\text{ finite}} \sum_{j \in S} f_j f_j^*
        \ = \ \sum_j f_j f_j^*
 \end{equation*}
and so~$\| f \|^2 = \|f^*\|^2 = \|\langle f,f \rangle \|
                = \|\sum_j f_j f_j^* \|$ as desired.
Finally, to show uniqueness,
    assume~$f'\colon A \to B$ is any (other)
    morphism with~$f' \kappa_i = f_i$.
    Note that~$\sum_j \kappa_j \kappa_j^*$
       converges ultrastrongly
       and so ultranorm as well.
With similar reasoning as before,
        the composition map~$g \mapsto f' g$ is ultranorm continuous,
        hence
\begin{equation*}
    f'
    \ =\ f' \sum_{j \in I} \kappa_j \kappa_j^*
    \ =\ \sum_{j \in I} f' \kappa_j \kappa_j^*
    \ =\ \sum_{j \in I} f_j \kappa_j^*
    \ =\ \sum_{j \in I} f \kappa_j \kappa_j^*
    \ =\ f,
\end{equation*}
    where the sums converge ultranorm.
\end{proof}

\begin{rem}
For finite $I$, these direct sums are also dagger limits in the sense of~\cite{daglims}. For infinite $I$, this is no longer the case, since infinite direct sums are not even limits; compare the degeneracy result~\cite[Theorem~5.2]{daglims}, which indicates that infinite dagger products cannot be expected to exist. Thus our Definition~\ref{directsumdef} seems better adapted to the context of Hilbert spaces or other W$^*$-categories than those of~\cite{daglims}.
\tob{We can tone down the last sentence a bit if it's too demeaning}
\end{rem}

As a simple consequence, we have a generalization of~\cite[Proposition~4.9]{Rieffel1974}.

\begin{cor}
A normal $*$-functor automatically preserves direct sums.
\end{cor}

\newcommand{\alg}{\cat{N}}	% a W$^*$-algebra

For a W$^*$-algebra $\alg$, we let $\NRep{\alg}$ be its category of normal representations on Hilbert spaces. For example, $\alg$ may be the double dual of a C$^*$-algebra, in which case $\NRep{\alg}$ is equivalent to the category of representations of the C$^*$-algebra.\footnote{Our choice of font $\alg$ is the same as that for categories, suggesting that $\alg$ may as well more generally be a small W$^*$-category, and the same statements would hold. Since one can always take the convolution W$^*$-algebra of any small W$^*$-category, resulting in the same category of representations, there is no loss of generality in restricting ourselves to W$^*$-algebras.}

\begin{thm}
In $\NRep{\alg}$, any $\ell^2$-direct sum
\[
	\bigoplus A_i := \left\{ (\xi_i)_{i\in I} \Biggm \vert \xi_i\in A_i,\: \sum_i \langle\xi_i,\xi_i\rangle < \infty \right\},
\]
equipped with the blockwise representation of $\alg$, is indeed an \emph{$I$-indexed direct sum} in the sense of Definition~\ref{directsumdef}. Every $I$-indexed direct sum in $\NRep{\alg}$ is of this form.
\end{thm}

\begin{proof}
\todo[inline]{do this}
\end{proof}

\section{The universal property of direct integrals}

\newcommand{\Obj}[1]{\mathrm{Obj}(#1)}
\newcommand{\Mor}[1]{\mathrm{Mor}(#1)}

\newcommand{\LMeas}{\cat{LocMeas}} 	% category of localizable measure spaces
\newcommand{\Null}{\mathcal{N}} 	% null sets

\tob{The precise definitions are not set in stone, and should be adjusted such that the measure theory works out. So far it's more of a sketch}

\subsection{Preliminaries}

We begin with some measure-theoretic preliminaries, following Fremlin~\cite{fremlin2,fremlin3}.

Given a measure space $(X,\Sigma,\mu)$, the \emph{null sets} are the sets of zero measure, and the \emph{full sets} are their complements. The measure space is \emph{localizable} if $\Sigma$ modulo the null sets is complete as a Boolean algebra, which means that every collection of measurable sets has an \emph{essential supremum}. Localizable measure spaces were introduced by Segal \cite{segal} as a characterization of the measure spaces such that $L^\infty(X)$, the C$^*$-algebra of essentially bounded measurable functions modulo null sets, is a W$^*$-algebra. 

With $\Null$ denoting the $\sigma$-ideal of null sets, the measure descends to a countably additive map $\Sigma / \Null \to \Rplusext$, such that every non-zero element of $\Sigma / \Null$ has non-zero measure. As $\Sigma / \Null$ is a complete Boolean algebra, such a map is necessarily completely additive, not just countably additive \cite[321E]{fremlin2}.

\tob{So we can e.g. take the measurable space $(S,2^S)$ for some uncountable $S$, and equip it with the measure $\mu$ which vanishes on any countable set, and is $\infty$ on all uncountable sets. This is a localizable measure space, right? I guess that it's not compact though, and clearly not strictly localizable. Is this a good (counter-)example for us?}

Unfortunately, there exist localizable measure spaces $X$, $Y$ such that $L^\infty(X) \cong L^\infty(Y)$ but there is no measurable map from $X$ to $Y$ \todo{example?}. To get out of this difficulty, we restrict ourselves to localizable measure spaces arising from \emph{complete strictly localizable compact} measure spaces. 

To define these notions, we first define the finite intersection property, and what a compact class is. Let $X$ be a set, and $\Kay \subseteq \powerset(X)$ be a family of sets. We say it has the \emph{finite intersection property} if for each finite subfamily $(K_i)_{i \in I}$, $K_i \in \Kay$, we have $\bigcap_{i \in I} K_i \neq 0$. We say that $\Kay$ is a \emph{compact class} if for any subfamily $\Kay' \subseteq \Kay$ having the finite intersection property, $\bigcap \Kay' \neq \emptyset$. For example, the closed sets of any compact topological space form a compact class. Slightly less trivially, the compact subsets of any Hausdorff space form a compact class. 

A measure space $(X,\Sigma,\mu)$ is \emph{compact} \cite[342A(c)]{fremlin3} if there exists a compact class $\Kay$ such that for all $S \in \Sigma$
\[
\mu(S) = \sup \{ \mu(K) \mid K \in \Kay \cap \Sigma \text{ and } K \subseteq S \}.
\]
This condition is known as being \emph{inner regular with respect to} $\Kay$. Therefore any Radon measure on a Hausdorff space $X$ is compact, because we can take $\Kay$ to be the set of (topologically) compact subsets of $X$. So familiar examples, such as Lebesgue measure on $\R^n$, the Haar measure on a locally compact group, or any $\sigma$-finite measure on a standard Borel space, are compact.

A measure space $(X,\Sigma,\mu)$ is \emph{strictly localizable} if it can be written as the (typically infinite) disjoint union of finite localizable measure spaces~\cite[211E]{fremlin2}.

We write $\LMeas$ for the category where:
\begin{enumerate}
\item Objects are measure spaces which are complete, strictly localizable and compact;
\item Morphisms are equivalence classes of measurable maps $f : X \to Y$ with $\mu_Y \ll f_* \mu_X$ \todo{or something like truly continuous?}, where for measurable $f,g : X \to Y$, we put $f \sim g$ iff for all $T \in \Sigma_Y$, we have $f^{-1}(T) \triangle g^{-1}(T) \in \Null_X$. \todo{Do we also want some compatibility with the compact classes?}
\end{enumerate}

If $f$ and $g$ agree almost everywhere, then $f \sim g$. The converse also holds if $(Y,\Sigma_Y)$ is countably separated \cite[343F]{fremlin3}, but not in general \cite[343I]{fremlin3}. 


\subsection{Measurable families of objects}

Now let $\cat{C}$ be a W$^*$-category, and $X \in \LMeas$. Then we define a category $L^\infty(X,\cat{C})$ as follows:

\begin{enumerate}
\item Objects are maps $A : X \to \cat{C}, \: x\mapsto A_x$, which are measurable in the sense that there is a measurable partition of $X$ such that $A$ is constant on each part;
\item Morphisms $f : A\to B$ are equivalence classes of bounded families $(f_x : A_x\to B_x)_{x\in X}$, which depend measurably \todo{work out what exactly this should mean such that composition makes sense} on $x$, and where $f \sim g$ if upon restriction to any measurable set on which both $A$ and $B$ are constant, we have $f^{-1}(T) \triangle g^{-1}(T) \in \Null_X$ for every measurable $T \subseteq \cat{C}(A,B)$.
\end{enumerate}

In particular, this means that two objects $A$ and $B$ which differ only on a null set are isomorphic.

\begin{prop}
$L^\infty(X,\cat{C})$ is a W$^*$-category as well.
\end{prop}

So the categories $L^\infty(X,\Hilb)$ are our version of Yetter's \emph{categories of measurable fields of Hilbert spaces}.

\todo[inline]{Can we make the comparison more precise?}

\begin{proof}
It is clear that $L^\infty(X,\cat{C})$ inherits the antilinear involution from $\cat{C}$. The norm of a morphism is defined as
\[
	\| (f_x) \| := \esssup_x \| f_x \|,
\]
which is clearly well-defined upon changing the representative. To show that each hom-set is also complete as a normed space, let $( (f_x)^{(n)})_{n\in\N}$ be a Cauchy sequence of (representatives of) morphisms. We show that the collection of all $x$ for which $(f_x^{(n)})_{n\in\N}$ is Cauchy is a full set. Since countable intersections of full sets are full, It is enough to prove that for every $\eps > 0$, the set of $x$ for which $\| f_x^{(n)} - f_x^{(m)} \| < \eps$ holds eventually is a full set. But by assumption, we know that $\esssup_x \| f_x^{(n)} - f_x^{(m)} \| < \eps$ eventually, which is enough. Therefore we can define a candidate representative of a limit morphism as $(f_x)$, where $f_x$ is the above limit on those $x$ where the family is Cauchy, and e.g.~zero outside of that. It is straightforward to show that this is the limit.

The submultiplicativity of the norm and the C$^*$-identity follow directly from those of $\cat{C}$.

\todo[inline]{Try to construct preduals for the hom-sets in terms of spaces like $L^1(X,\cat{C}(A,B)_*)$?}
\end{proof}


For $(A_x)_{x\in X}$ a measured family of objects in $\cat{C}$, we now consider the functor
\[
	\int_X \cat{C}(A_x,-) \, d\mu(x) \: : \: \cat{C}\to\Set
\]
given by associating to every object $B$ the set
\[
	\int_X \cat{C}(A_x , B) \, d\mu(x) := \left\{ f : X\to\Mor{\cat{C}} \textrm{ measurable} \Biggm\vert f_x : A_x\to B,\: \int_X f_x f_x^* \,dx < \infty \right\},
\]
which is functorial in $B$ in the obvious way.

\begin{defn}
A \emph{direct integral} of the family $(A_x)_{x\in X}$ is now an object $\int_X A_x \, d\mu(x)$ together with a natural bijection
\[
	\cat{C}\left( \int_X A_x \, d\mu(x), -\right) \cong \int_X \cat{C}(A_x,-) \, d\mu(x).
\]
\end{defn}

\todo[inline]{Is this the ``right'' definition? What is the relation to~\cite{measureablecats}?}

\begin{defn}
A W$^*$-category \emph{has direct integrals} if it has direct integrals for $\sigma$-finite measure spaces and arbitrary direct sums.
\end{defn}

\todo[inline]{Is the restriction to $\sigma$-finite the ``right'' definition?}

Due to the existence of countable direct sums, we could as well restrict to direct integrals on finite measure spaces:

\begin{prop}
If countable direct sums and direct integrals for finite measure spaces and exist, then so do direct integrals for $\sigma$-finite measure spaces.
\end{prop}

If everything works out, this is due to a nice kind of compositionality of direct integrals (essentially Fubini's theorem).

\todo[inline]{Prove Fubini's theorem for direct integrals from the universal property?}

\begin{thm}
Suppose that $\cat{C}$ has direct integrals. Then every $\int_X A_x \, d\mu(x)$ carries a normal representation of $L^\infty(X)$, and we have a functor
\[
	L^\infty(X,\cat{C}) \to \NRep{L^\infty(X),\cat{C}} , \qquad (A_x) \mapsto \int_X A_x \, d\mu(x),
\]
and this is an equivalence of W$^*$-categories.
\end{thm}


\bibliographystyle{plain}
\bibliography{direct_integrals}

\end{document}

% vim: ft=tex.latex
