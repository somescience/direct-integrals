\documentclass[reqno,T1]{amsproc}
\usepackage{amssymb}
\usepackage{amsmath}
\usepackage{amsfonts}
\usepackage[canadian]{babel}
\usepackage{xcolor}
\usepackage{geometry}
\usepackage{xifthen} % provides \isempty
\usepackage{xparse} % new commands with more than one optional argument

% commutative diagrams
\usepackage{tikz}
\usetikzlibrary{arrows,cd,decorations.markings,shapes.geometric,shapes}

% color of links
\definecolor{myurlcolor}{rgb}{0,0,0.4}
\definecolor{mycitecolor}{rgb}{0,0.5,0}
\definecolor{myrefcolor}{rgb}{0.5,0,0}
\usepackage[pagebackref,draft=false]{hyperref}
\hypersetup{colorlinks,
linkcolor=myrefcolor,
citecolor=mycitecolor,
urlcolor=myurlcolor}
\renewcommand*{\backref}[1]{$\uparrow$\,#1}

% general mathematical macros
\renewcommand{\H}{\mathcal{H}}	% Hilbert space
\newcommand{\beq}{\begin{equation}}
\newcommand{\eeq}{\end{equation}}
\newcommand{\N}{\mathbb{N}}
\newcommand{\Z}{\mathbb{Z}}
\newcommand{\Q}{\mathbb{Q}}
\newcommand{\R}{\mathbb{R}}
\newcommand{\C}{\mathbb{C}}
\newcommand{\op}{\mathrm{op}}
\newcommand{\eps}{\varepsilon}
\newcommand{\lin}[1]{\mathrm{lin}(#1)}
\newcommand{\dom}{\mathrm{dom}}
\newcommand{\supp}{\mathrm{supp}}

% categories
\newcommand{\cat}[1]{\mathsf{#1}}		% generic category
\newcommand{\Set}{\mathsf{Set}}
\newcommand{\Vect}[1]{\mathsf{Vect}_{#1}}	% vector spaces over a field
\newcommand{\Ban}{\mathsf{Ban}}			% (complex) Banach spaces
\newcommand{\Hilb}{\mathsf{Hilb}} 		% Hilb, cat of Hilbert spaces
\newcommand{\Rep}[1]{\mathsf{Rep}(#1)}		% cat normal Hilbert space reps of a C*-algebra
\newcommand{\NRep}[1]{\mathsf{NRep}(#1)}	% cat of normal Hilbert space reps of a W*-algebra
\newcommand{\id}[1]{1_{#1}}			% identity morphism

% Theorem Environments
\swapnumbers
\theoremstyle{plain}
\newtheorem{thm}{Theorem}[section]
\newtheorem{lem}[thm]{Lemma}
\newtheorem{prop}[thm]{Proposition}
\newtheorem{cor}[thm]{Corollary}
\newtheorem{conj}[thm]{Conjecture}
\newtheorem{qstn}[thm]{Question}
\newtheorem{defn}[thm]{Definition}
\newtheorem{lemdefn}[thm]{Lemma + Definition}
\newtheorem{prob}[thm]{Problem}
\theoremstyle{remark}
\newtheorem{ex}[thm]{Example}
\newtheorem{rem}[thm]{Remark}
\newtheorem{note}[thm]{Note}
\numberwithin{equation}{section}

%%%%%%%%%%%% Enumeration via lowercase letters
\renewcommand{\labelenumi}{(\alph{enumi})}
\renewcommand{\theenumi}{(\alph{enumi})}
\renewcommand{\labelitemi}{$\circ$}

%%% pagebreaks allowed for align environment
\allowdisplaybreaks

% string diagram stuff
\pgfdeclarelayer{edgelayer}
\pgfdeclarelayer{nodelayer}
\pgfsetlayers{edgelayer,nodelayer,main}
\tikzstyle{none}=[inner sep=0pt]
\tikzstyle{simple}=[-,draw=black,line width=1.000]
\tikzstyle{bn}=[circle,inner sep=2pt,fill=black,draw=black,line width=0.8 pt]

%%% draft stuff
%\usepackage[draft]{showkeys}
\usepackage{todonotes}
\newcommand{\tob}[1]{\todo[color=blue!40,inline]{#1}}


%-------------------------------------------------------------------

\begin{document}
\sloppy

% vertical spacing in multiline equations
\setlength{\jot}{6pt}

%-------------------------------------------------------------------

%%%%%%%%%%%% title page stuff %%%%%%%%%%%%%%%%%%%%%%%%%%

\title{The universal property of direct integrals in W*-categories}

\author{Tobias Fritz}

\address{Max Planck Institute for Mathematics in the Sciences, Leipzig, Germany}
\email{fritz@mis.mpg.de}

\keywords{}

\subjclass[2010]{Primary: ; Secondary: }

\thanks{\textit{Acknowledgements.} }

\begin{abstract}
We formulate a universal property for infinite direct sums in W*-categories, and prove that this indeed characterizes the usual infinite direct sums in the W*-category of representations of a C*-algebra. We do the same for direct integrals, in which case we need to work with measurable W*-categories.
\end{abstract}

\maketitle

\section{Introduction}

\section{W*-categories}

Let $C$ be a W*-category~\cite{wstarcat}, such as $\Hilb$ or $\Rep{A}$, the category of Hilbert space representations of a C*-algebra $A$.

\begin{lem}
Let $u : A \to B$ be an invertible morphism with $\| u \| = \| u^{-1} \| = 1$. Then $u$ is unitary.
\label{unitaries}
\end{lem}

\todo[inline]{We could state this at the generality of C*-cats}

\begin{proof}
We have $\| u^* u \| = 1$ by the C*-identity, and likewise $\| u^{-1} (u^{-1})^* \| = 1$. Since $(u^* u)^{-1} = u^{-1} (u^{-1})^*$, we also have $\| (u^* u)^{-1} \| = 1$. This makes $u^* u$ into a positive element of unit norm whose inverse also has unit norm, and therefore $u^* u = \id{A}$. By the same token, we obtain $u u^* = \id{B}$.
\end{proof}

Now let $\Ban$ be the category of Banach spaces with bounded linear maps of norm $\leq 1$. Every W*-category is $\Ban$-enriched. Lemma~\ref{unitaries} and the Yoneda lemma imply that for any representable functor $\cat{C} \to \Ban$, the representing object is unique up to unique \emph{unitary} isomorphism. This is a desideratum of a notion of universal property for W*-categories, and more generally for all dagger categories~\cite{daglims}. The lemma tells us that the compatibility with the dagger structure is automatic thanks to the enrichment in $\Ban$, so that we do not need to formulate our universal properties in the form of dagger limits as in~\cite{daglims}.

\section{The universal property of infinite direct sums}

For $(A_i)_{i\in I}$ a family of objects in $\cat{C}$, we consider the functor
\[
	\bigoplus_i \cat{C}(A_i,-) \: : \: \cat{C} \to \Ban
\]
given by associating to every object $B$ the vector space
\[
	\bigoplus_i \cat{C}(A_i , B) := \left\{ (f_i : A_i \to B)_{i\in I} \Biggm\vert \sum_i f_i f_i^* < \infty \right\},
\]
considered as a Banach space under the norm
\[
	\| (f_i)_{i \in I} \| := \left\| \sum_i f_i f_i^* \right\|^{1/2}.
\]
This is functorial in $B$ in the obvious way.

\todo[inline]{Is it actually complete?}

\begin{defn}
\label{directsumdef}
An \emph{$I$-indexed direct sum} of the family of objects $(A_i)_{i\in I}$ is an object $\bigoplus_i A_i$ which represents the $\Ban$-enriched functor $\bigoplus_i \cat{C}(A_i,-)$.
\end{defn}

By Lemma~\ref{unitaries} and the $\Ban$-enriched Yoneda lemma, this characterizes infinite direct sums up to unique isomorphism. Since we are in a dagger category, the same holds dually, meaning that $\bigoplus_i A_i$ also represents the functor
\[
	\bigoplus_i \cat{C}(-,A_i) \: : \: \cat{C}^\op \to \Ban
\]
given by associating to every object $B$ the vector space
\[
	\bigoplus_i \cat{C}(B, A_i) := \left\{ (f_i : B \to A_i)_{i\in I} \Biggm\vert \sum_i f_i^* f_i < \infty \right\},
\]
with norm
\[
	\| (f_i)_{i \in I} \| := \left\| \sum_i f_i^* f_i \right\|^{1/2}.
\]
which is functorial in $B$ in the obvious way.

\newcommand{\alg}{N}	% a W*-algebra

For a W*-algebra $\alg$, we let $\NRep{\alg}$ be its category of normal representations on Hilbert spaces. For example, $\alg$ may be the double dual of a C*-algebra, in which case $\NRep{\alg}$ is equivalent to the category of representations of the C*-algebra.\footnote{Our choice of font $\alg$ is the same as that for categories, suggesting that $\alg$ may as well more generally be a small W*-category, and the same statements would hold. Since one can always take the convolution W*-algebra of any small W*-category, resulting in the same category of representations, there is no loss of generality in restricting ourselves to W*-algebras.}

\begin{thm}
In $\NRep{\alg}$, any $\ell^2$-direct sum
\[
	\bigoplus A_i := \left\{ (\xi_i)_{i\in I} \Biggm \vert \xi_i\in A_i,\: \sum_i \langle\xi_i,\xi_i\rangle < \infty \right\},
\]
equipped with the blockwise representation of $\alg$, is indeed an \emph{$I$-indexed direct sum} in the sense of Definition~\ref{directsumdef}. Every $I$-indexed direct sum in $\NRep{\alg}$ is of this form.
\end{thm}

\begin{proof}
\todo[inline]{do this}
\end{proof}

\begin{thm}
The following are equivalent:
\begin{enumerate}
\item $\bigoplus_i A_i$ is an $I$-indexed direct sum.
\item There is a family of morphisms $\left(\kappa_j : A_j \to \bigoplus_i A_i\right)_{j\in I}$ such that $\kappa_j^*\kappa_k = \delta_{jk}$, and
\[
	\sum_j \kappa_j \kappa_j^* = 1_{\bigoplus_i A_i}
\]
ultraweakly.
\end{enumerate}
\end{thm}

\begin{proof}
\todo[inline]{do this}
\end{proof}

\todo[inline]{Does this also result in a dagger limit in the sense of~\cite{daglims}?}

\section{Direct integrals}

\newcommand{\Obj}[1]{\mathrm{Obj}(#1)}
\newcommand{\Mor}[1]{\mathrm{Mor}(#1)}

\begin{defn}
A \emph{measurable category} $\cat{C}$ is a category internal to measurable spaces. A \emph{measured family of objects} indexed by a measure space $(X,\Sigma,\mu)$ is a measurable function $A : X\to\Obj{\cat{C}}$.
\end{defn}

For $(A_x)_{x\in X}$ a measured family of objects in $\cat{C}$, we now consider the functor
\[
	\int_X \cat{C}(A_x,-) \: : \: \cat{C}\to\Set
\]
given by associating to every object $B$ the set
\[
	\int_X \cat{C}(A_x , B) := \left\{ f : X\to\Mor{\cat{C}} \textrm{ measurable} \Biggm\vert f_x : A_x\to B,\: \int_X f_x f_x^* \,dx < \infty \right\},
\]
which is functorial in $B$ in the obvious way.

\todo[inline]{Is this the ``right'' definition? What is the relation to~\cite{measureablecats}?}

\begin{defn}
A W*-category \emph{has direct integrals} if it has direct integrals for $\sigma$-finite measure spaces and arbitrary direct sums.
\end{defn}

\todo[inline]{Is the restriction to $\sigma$-finite the ``right'' definition?}

Due to the existence of countable direct sums, we could as well restrict to direct integrals on finite measure spaces:

\begin{prop}
If countable direct sums and direct integrals for finite measure spaces and exist, then so do direct integrals for $\sigma$-finite measure spaces.
\end{prop}

If everything works out, this is due to a nice kind of compositionality of direct integrals (essentially Fubini's theorem).

\todo[inline]{Prove Fubini's theorem for direct integrals from the universal property?}

\bibliographystyle{plain}
\bibliography{direct_integrals}

\end{document}
